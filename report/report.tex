\documentclass[11pt,a4paper]{scrartcl}
\usepackage[latin1]{inputenc}
\usepackage{amsmath}
\usepackage{amsfonts}
\usepackage{amssymb}
\usepackage{graphicx}
\usepackage{longtable}
\usepackage{array}

\addtokomafont{disposition}{\rmfamily}



\title{MSc, MEng and MMath Examinations, 2017?18
	DEPARTMENT OF COMPUTER SCIENCE
	Model-Driven Engineering (MODE)
	Open Individual Assessment}
\date{}

\begin{document}
	%-----------------------------------------------------
	% Title Page
	%-----------------------------------------------------
	
	\begin{titlepage}
		\begin{center}
			
			\large
			\textbf{University of York} \\
			\vspace{0.2cm}
			\textbf{MSc, MEng and MMath Examinations, 2017-18} \\
			\vspace{0.2cm}
			\textbf{DEPARTMENT OF COMPUTER SCIENCE} \\
			
			\vspace{2cm}
			
			\huge
			\textbf{Model-Driven Engineering (MODE)} 
			\textbf{Open Individual Assessment} \\
			
			\vspace{2cm}
			
			\Large
			\textbf{Examination Number} \\
			\textbf{Y1403115}
			
		\end{center}
	\end{titlepage}
	
	


%In this DSL, a requirements model comprises a number of requirements, test-cases
%and team members. There are two types of requirements: customer requirements and
%system requirements. Each requirement has an identifier and a textual description.
%Requirements can be hierarchically decomposed (e.g. requirement A cannot be satis-
%fied, unless requirements B and C are satisfied) and can also be in conflict with other
%requirements (e.g. only one of requirements A and B can be satisfied). Each test-case
%has a description and can be used to verify one or more technical requirements. Some
%additional properties include:
%? The identifier of each requirement is unique
%? The textual description of a requirement is at least 10 characters long
%? Customer requirements cannot be in conflict with system requirements (and
%vice-versa)
%? Customer requirements can be decomposed into technical requirements, but not
%the other way round
%? Each customer requirement needs to be decomposed into at least one technical
%requirement
%? Each test-case needs to be verifying at least one technical requirement
%? Each technical requirement needs originate from at least one customer require-
%ment
%? Each leaf requirement (i.e. requirement that is not further decomposed) has a
%"progress" field that records how much progress the development team has made
%towards satisfying it so far (e.g. 50%)
%? The progress of a non-leaf requirement is the average of the progress of the
%requirements it is decomposed into (e.g. if requirement A is decomposed into
%requirements B and C, where B and C are at 10% and 20% complete respectively,
%the progress of A is 15%)
%? Each requirement can be allocated to one or more team members
%? Requirements decomposition must be free of cycles



\section{Abstract Syntax and Constraints}
% Explain how you have arrived at the abstract syntax and any well-formedness constraints for your DSL, justify the decisions that you have made, and discuss how you have tested your abstract syntax and well-formedness constraints.



\begin{center}
	
	\begin{longtable}{| p{0.5\linewidth} | p{0.5\linewidth} |}
	\hline
	\textbf{User Requirement} & \textbf{System Requirement} \\ \hline
	
	Requirements 
	& Requirements can be two types User and System. \\ \hline
	& Requirements should have a unique identifier. \\ \hline
	& Requirements should have a progress  field. \\ \hline
	& Requirements should have description which is at least 10 characters. \\ \hline
	
	Requirements Conflicts 
	& Customer Requirements can be in conflict with other Customer Requirements 0..* \\ \hline
	& System Requirements can be in conflict with other System Requirements 0..* \\ \hline
	
	Requirements Decomposition 
	& Each customer requirement is decomposed into 1..* System Requirements\\ \hline
	& Each System requirement needs to originate from 1..* User Requirements. \\ \hline
	& Each Requirement(it has to be System) that cannot be decomposed has a progress field \\ \hline
	& Each Requirement that can be decomposed further has progress = average progress of the requirements it decomposes to. \\ \hline
	& Requirements decomposition must be free of cycles. \\ \hline
	
	Test Cases
	& Each test case has a description \\ \hline
	& Each test case verifies 1..* System Requirements \\ \hline
	
	Team Members
	& Each Requirement can be allocated to 1..* Team Members\\ \hline
	
	
	\hline
\end{longtable}
\end{center}	




System requirements cannot be root nodes.

Customer requirements cannot be leaf nodes.

System requirements cannot be parents of customer requirements.

\subsection{Stuff eliminated by metamodel}
Customer requirements cannot be leaf nodes because they need to have at least one child.

\section{Editor}
% Explain the editor for your DSL, justify the decisions that you have made, and briefly argue why your editor provides an appropriate notation for your DSL.

\subsection{Well formdness constraints}
\begin{enumerate}
	\item Requirements
	\begin{enumerate}
		\item Requirement identifier should be unique.
		\item Requirement description should be at least 10 characters long.
		\item Progress must be between 0 and 100 \%.
		\item Conflicting requirements with common ancestor should not both contribute to progress average of ancestor.
		
		\item Requirements cannot be in conflict and dependency at the same time.
		
		\item There cannot be any loops in dependencies including self referential dependencies.
		
		\item System Requirements cannot be parents of Customer Requirements.
		\item System Requirements cannot be root nodes.
		\item Customer Requirements cannot be leaf nodes.
	\end{enumerate}
	
	\item Team members
	\begin{enumerate}
		\item Should have a unique user name.
		\item Requirements allocated to team members include all child requirements.
		\item Requirements cannot be allocated to team members twice.
	\end{enumerate}
	\item Test cases
	\begin{enumerate}
		\item Test cases identifier is unique.
		\item Test cases verify only system requirements.
		\item Each test case verifies at least on system requirement.
	\end{enumerate}
\end{enumerate}











\section{Model Management Operations}
%Explain the model management operations for your DSL, justify the decisions that you have made, and briefly demonstrate that the model management operations can be used to fulfill the model management tasks described in Section 1.

\pagebreak
\appendix
\section{Appendix A}
\begin{center}
	\begin{longtable}{| p{0.5\linewidth} | p{0.5\linewidth} |}
		\hline
		\textbf{Scenario} & \textbf{Constraint} \\ \hline
		
		In this DSL, a requirements model comprises a number of requirements, test-cases and team members. 
		& \\ \hline
		
		There are two types of requirements: customer requirements and system requirements. 
		& Customer requirement and system requirements should be subclasses of requirements. \\ \hline
		
		Each requirement has an identifier and a textual description.
		& All requirements have an id and description fields \\ \hline
		
		Requirements can be hierarchically decomposed (e.g. requirement A cannot be satisfied, unless requirements B and C are satisfied) and can also be in conflict with other requirements (e.g. only one of requirements A and B can be satisfied). 
		& Requirements should be part of a tree like structure where each requirement is dependant on it's child requirements. Requirements can be in conflict with other requirements meaning that only one or the other can be satisfied. \\ \hline
		
		Each test-case has a description and can be used to verify one or more technical requirements.
		& Test cases have a description field and verify 1 or more requirements. \\ \hline
		The identifier of each requirement is unique
		& \\ \hline
		The textual description of a requirement is at least 10 characters long
		& \\ \hline
		Customer requirements cannot be in conflict with system requirements (and vice-versa)
		& \\ \hline
		Customer requirements can be decomposed into technical requirements, but not the other way round
		& \\ \hline
		Each customer requirement needs to be decomposed into at least one technical requirement
		& \\ \hline
		Each test-case needs to be verifying at least one technical requirement
		& \\ \hline
		Each technical requirement needs originate from at least one customer requirement
		& \\ \hline
		Each leaf requirement (i.e. requirement that is not further decomposed) has a "progress" field that records how much progress the development team has made towards satisfying it so far (e.g. 50%)
		& \\ \hline
		The progress of a non-leaf requirement is the average of the progress of the requirements it is decomposed into (e.g. if requirement A is decomposed into requirements B and C, where B and C are at 10\% and 20\% complete respectively, the progress of A is 15%)
		& \\ \hline
		Each requirement can be allocated to one or more team members
		& \\ \hline
		Requirements decomposition must be free of cycles
		& \\ \hline
		
		\hline
	\end{longtable}
\end{center}	

	
\end{document}